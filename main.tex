\documentclass{article}
\usepackage[utf8]{inputenc}

\title{Proposta de priorização de áreas de restauração na Bacia Hidrográfica do Rio Doce: Como usar os resultados de Strassburg et al. 2019?}
\author{Grupo NCCG-JBRJ}
\date{March 2020}

\usepackage{biblatex}
% Eu vi que o Overleaf tem a opção de conectar o .bib com o Mendeley, mas é só na versão paga. Acho que vamos ter que ir atualizando o .bib manualmente, quando necessário
\addbibresource{refs_proj2PCI_20-03-2020.bib}
\usepackage{graphicx}

\begin{document}

\maketitle

\section{Breve descrição do estudo}
O estudo de Strassburg et al. \cite{Strassburg2019} investigou cenários alternativos para a restauração ecológica na Mata Atlântica. Os autores utilizaram a abordagem de priorização baseada em programação linear. A priorização foi baseada em três cenários básicos (Figura \ref{fig:results}): 

\begin{itemize}
    \item Biodiversidade: Diminuição do risco  de extinção associado ao aumento de áreas de habitat adequados para cada espécie
    \item Mudanças climáticas: Aumento do sequestro de carbono em áreas degradadas
    \item Custos: Valor gasto em restauração, incluindo o potencial de perda de receita da agricultura ou pecuária de áreas que estão sendo restauradas
\end{itemize}
 
Além dos três anteriores, foi proposto um cenário "final", combinando o melhor possível dos três objetivos: maximização da biodiversidade, maximização de sequestro de carbono e diminuição de custos (Figura \ref{fig:results}).

\section{Como poderia ser aplicado no nosso primeiro produto}
Inicialmente foi feito um recorte simples do cenário final de Strassburg et al. \cite{Strassburg2019} para a área da Bacia Hidrográfica do Rio Doce (BHRD, Figura \ref{fig:scenarios_BHRD}).

No entanto, este simples recorte gera um erro de interpretação nos valores dos pixels, por causa da diferença de escala. No resultado original de Strassburg et al. \cite{Strassburg2019}, os valores representam quantas vezes o pixel foi considerado prioritário em relação aos demais pixels da área de estudo (toda a Mata Atlântica). Portanto, esta "prioridade relativa" não faz sentido somente para a área da BHRD.
%ast. nao sei se para o texto, mas para analisar, a gente precisa relembrar as camadas que foram usadas nessa priorizacao do iis. camada biotica ok, endemicas com mais de 10 registros, modeladas com 3 algoritmos. o resto foi custo de orpotunidade mesmo e mais o que? 
Sugestão: Descrever as características particulares da BHRD relevantes para a tomada de decisão em restauração (biodiversidade, uso do solo incluindo mineração, áreas protegidas) e, posteriormente, discutir/comparar com a análise em macroescala, na Mata Atlântica, feita por Strassburg et al. \cite{Strassburg2019}.

\section{Proposta para o primeiro produto}

Perguntas para guiar a execução do primeiro produto (divisão preliminar de tarefas):

\begin{itemize}
    \item Biodiversidade:
    \begin{enumerate}
        \item Quais são as espécies vegetais de ocorrência na BHRD? 
        \item Quais delas são endêmicas?
        \item Como as espécies são distribuídas nas microbacias? 
        \item Quais são as áreas de maior riqueza na BHRD?
    \end{enumerate}
    
    \item Uso do solo:
    \begin{enumerate}
        \item Como se distribuem as diferentes classes de uso do solo na BHRD? 
        \item Essa distribuição é uniforme ao longo da BHRD ou varia nas diferentes microbacias? Como varia?
        \item Quais microbacias possuem maior cobertura vegetal?
        \item Qual é a proporcão da área ocupada por cada uma delas na bacia?
        \item ...
    \end{enumerate}
    
    \item Mineração:
    \begin{enumerate}
        \item Qual é a porcentagem da área total da BHRD usada para mineração?
        \item Quantas destas áreas estão em atividade? Quantas estão planejadas?
        \item Como a área destinada a mineração na BHRD mudaria com as PLs propostas: PL 37/2011 (propõe permitir mineração em UCs de uso sustentável), PL 3682/2012 (permite mineração em até 10\% da área de UCs de proteção integral) e PL 1610/1996 (permite mineração em terras indígenas).
    \end{enumerate}
    
    \item Áreas protegidas:
    \begin{enumerate}
        \item Qual é a porcentagem da área total da BHRD já protegida por UCs?
        \item Qual o tamanho das UCs? São fragmentos grandes ou pequenos e como estes fragmentos (UCs) estão conectados entre si?
        \item Como é a distribuição de UCs nas diferentes microbacias? 
        \item Como é o entorno dessas UCs?
    \end{enumerate} 

    \item Áreas para restauração:
    \begin{enumerate}
        \item Baseando-se nas respostas das perguntas anteriores, conseguimos apontar áreas "receptíveis" à restauração? Não sei se faz sentido, mas seria uma primeira avaliação, antes da priorização, que dependerá dos modelos das espécies (segundo produto?).
        \item 
    \end{enumerate} 
\end{itemize}


\begin{figure}[h!]
\centering
\includegraphics[scale=0.6]{figs/main_results.png}
\caption{Cenários de restauração ecológica na Mata Atlântica \cite{Strassburg2019}}
\label{fig:results}
\end{figure}

\begin{figure}
    \centering
    \includegraphics[scale=0.7]{figs/scenarios_BHRD.png}
    \caption{Cenários de Strassburg et al. \cite{Strassburg2019} cortados para a área da Bacia Hidrográfica do Rio Doce}
    \label{fig:scenarios_BHRD}
\end{figure}

\section{Requisitos e alternativas para uma priorização no âmbito do PCI}%ast srm
Um cenário de priorização da conservação ou da restauração na escala da Bacia vai precisar dos seguintes elementos: 

\begin{itemize}
    \item A definição do objetivo da priorização: conservação ou restauração? e levando em conta quais premissas? O planejamento estratégico para conservação foi desenhado inicialmente no contexto da conservação e adaptações posteriores são as que têm falado de restauração (inclusive Strassburg) %ast isso também sou eu pegando o bonde andando, acho.
    \item O levantamento e a definição dos dados que serão utilizados. Basicamente camada(s) biótica(s) e camadas de variáveis abióticas -mas quais e como - a lista elencada na proposta está boa, mas é preciso detectar se falta alguma variável essencial para responder a pergunta ou se sobra alguma. Algumas destas variáveis que faltam não são nada triviais de se obter e podem requerir que a gente faça uma parceria p.ex. com o IIS, seja com eles facilitando código e cálculos que lhes permitem gerar essas camadas (econômicas, estoque de carbono, custo de oportunidade), ou idealmente gerando essas camadas, com a extensão e resolução que o projeto definir. %A gente acha (srm, ast) que esta fase é a única que pode ser de fato implementada em um curto período de tempo mas elencar isto bem é crucial para qualquer produto futuro. 
    \item A escolha do método mais apropriado para a priorização. Fora softwares comerciais clássicos (zonation, marxan) e scripts privados %uma pena que não sejam disponibilizados
    tem o pacote prioritizr de R que usa uma licença comercial para a optimização. 
    
\end{itemize}

\printbibliography
\end{document}
